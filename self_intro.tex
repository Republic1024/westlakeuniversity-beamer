\documentclass[11pt]{beamer}
\usetheme{Madrid}
\usefonttheme{serif}

\usepackage[utf8]{inputenc}
\usepackage[english]{babel}
\usepackage{xeCJK}
\setCJKmainfont{MiSans}

\usepackage{etoolbox}
\AtBeginEnvironment{frame}{\raggedright}

\usepackage{amsmath, amsfonts, amssymb, booktabs, graphicx, listings}
\usepackage{hyperref}

\title{Self-Introduction}
\author[Alex Chen]{\textbf{Alex Chen} \\ \small Westalke University · Computer Science}
\institute[]{\textit{Westalke University} \\ Department of Computer Science and Technology}
\date{Self-introduction Presentation}

\begin{document}

\begin{frame}
    \titlepage
    \begin{center}
      \includegraphics[width=5cm]{./pic/logo3.png}
    \end{center}
\end{frame}

\begin{frame}{Self Introduction | 自我介绍}
\begin{itemize}
  \item Name: Alex Chen(陈思远)
  \item Education: BSc in Computer Science and Technology (AI Track)
    \begin{itemize}
      \item Westalke University(西湖大学)
    \end{itemize}
  \item Current Status:
    \begin{itemize}
      \item Received MSc offer from FutureTech University
    \end{itemize}
  \item Personal Strengths
    \begin{itemize}
      \item Mathematical intuition: Developed early interest in similarity measures and optimization algorithms.
      \item Coding skills: Proficient in end-to-end model training and deployment.
      \item Oral Speaking: IELTS Oral 7.0
    \end{itemize}
\end{itemize}
\end{frame}

\begin{frame}{Technical Skills | 技术能力}
\begin{itemize}
  \item Programming Languages: Python (PyTorch, NumPy, Scikit-learn), C++, Rust, SQL, LaTeX
  \item Tools: PyTorch, Hugging Face, Weights \& Biases, Git
  \item Familiar Fields:
    \begin{itemize}
      \item Data Science
      \item Deep Learning Optimization
      \item LLM Applications
      \item Reinforcement Learning
      \item Computer Vision
    \end{itemize}
\end{itemize}
\end{frame}

\begin{frame}{Research Philosophy | 研究理念}
\begin{quote}
  “In deep learning, everything is a matrix; in data science, everything is a table.”
  \newline
  “在深度学习中,一切皆为矩阵;在数据科学中,一切皆为表格。”
\end{quote}
\begin{itemize}
  \item Start from minimal working demos, expand datasets, and iterate ideas.
  \item Apply functional programming and modular thinking to system design.
\end{itemize}
\end{frame}

\begin{frame}{Project 1 - Autonomous Robot Navigation 【Research Project】}
\begin{itemize}
  \item Built a multi-sensor fusion system for autonomous path planning.
  \item Integrated LiDAR, IMU, and vision data for robust navigation.
  \item Improved trajectory accuracy through dynamic obstacle avoidance.
\end{itemize}
\end{frame}

\begin{frame}{Project 2 - Urban Traffic Flow Prediction 【Competition】}
\begin{itemize}
  \item \textbf{Competition}: Global Urban Computing Challenge (2nd Prize)
  \item \textbf{Goal}: Predict future traffic conditions based on temporal graph networks.
  \item \textbf{Techniques}: Graph neural networks, time-series forecasting, edge-level regression.
\end{itemize}
\end{frame}




\begin{frame}{Project 3 - Knowledge Graph RAG for Scientific Literature Retrieval 【Competition】}
\textbf{🏆 Competition Background}
\begin{itemize}
  \item 10th Global Data Science Challenge (2023.12–2024.04)
\end{itemize}
\textbf{🔍 Project Overview}
\begin{itemize}
  \item Built a RAG-enhanced system to retrieve and reason over scientific papers.
  \item Designed an intuitive Chatbot interface for researchers.
  \item Integrated LLMs with structured knowledge bases for literature recommendation.
\end{itemize}
\end{frame}


\begin{frame}{Research Thinking – Idea Origin | 科研思维与灵感来源}
  \begin{itemize}
    \item Research ideas are often derived from practical system limitations and user needs.
    \item Core Logic: observation $\rightarrow$ hypothesis $\rightarrow$ implementation.
  \end{itemize}
  \end{frame}



\begin{frame}{Research Summary | 项目与技术总结}
\renewcommand{\arraystretch}{1.4}
\begin{tabular}{p{2cm} p{2cm} p{2.5cm} p{4cm}}
\toprule
项目 & 应用方向 & 核心技术 & 技术亮点 \\
\midrule
Robot Navigation & 自动驾驶 & LiDAR + SLAM & 动态避障与路径优化 \\
Traffic Prediction & 城市交通预测 & 图神经网络 & 时空建模与预测优化 \\
Knowledge RAG & 科研文献检索 & RAG + 知识图谱 & 文献关系挖掘与推荐 \\
\bottomrule
\end{tabular}
\end{frame}

\begin{frame}{Why Westalke University? | 为什么选择西湖大学}
\begin{itemize}
  \item Research directions align with my interests in AI and systems engineering.
  \item Interdisciplinary and innovative academic environment.
  \item Strong mentorship and collaborative research culture.
  \item Westalke is rapidly emerging as a center for scientific innovation.
\end{itemize}
\end{frame}

\begin{frame}{Future Goals | 未来目标}
\begin{itemize}
  \item Short-term: Conduct applied research and publish papers.
  \item Long-term: Pursue PhD studies and bridge academic research with real-world applications.
\end{itemize}
\end{frame}

\begin{frame}{Thank You | 致谢}
\begin{itemize}
  \item Looking forward to joining Westalke University.
  \item Contact:
    \begin{itemize}
      \item Email: \href{mailto:example@example.com}{example@example.com}
      \item GitHub: \url{https://github.com/ExampleUser}
    \end{itemize}
\end{itemize}
\end{frame}

\end{document}
